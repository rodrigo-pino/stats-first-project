\documentclass[10pt,twocolumn,a4paper]{article}
\usepackage[latin1]{inputenc}
\usepackage{amsmath}
\usepackage{amsfonts}
\usepackage{amssymb}
\usepackage{graphicx}
\usepackage{tikz}
\usepackage[width=17.00cm]{geometry}
\newcommand{\argt}{\theta}



\title{\LARGE{\textbf{FESTIVAL DE LA CLASE. MATCOM. 2020} \\
		
		
		
		 ASIGNATURA \; \; \; LIC MI CARRERA\\
	
	
	
	\textbf{Orientaciones metodol\'ogicas:\\} Este es el tema de mi clase\\}

Estudiante P\'erez P\'erez \footnote{Soy estudiante de X a�o de mi carrera.}\\}
\date{}
%\renewcommand{\baselinestretch}{1.4}


\newtheorem{eje}{Ejercicio}
\newcommand{\sen}{\mbox{sen \hspace{0.001cm}}}
\newcommand{\cis}{\hspace{0.5mm}\mbox{cis}\hspace{0.5mm}}
\newcommand{\real}{\mathbb{R}}
\newcommand{\complex}{\mathbb{C}}




\begin{document}
\maketitle
\setcounter{page}{1}


\section*{Ejercicios}
\begin{eje}
	Propuesta de distintos ejercicios de la clase, para desarrollar las habilidades a crear durante la clase.
\end{eje}
\begin{eje}
	Cada uno de los elementos a tratar es optativo, as\'i como la estructura de la composici\'on de la clase. El presente documento constituye solamente una gu\'ia para su desarrollo
\end{eje}


\section*{Objetivos}
\begin{enumerate}
	\item Esta secci\'on va dedicada a los objetivos de la clase, las metas para el encuentro y ciertas especificidades que considere de importancia resaltar durante el trancurso de la clase. 
	\item Seg\'un la tem\'atica se pueden hacer alusi\'on a los medios de ense�anza utilizados convenientemente.
\end{enumerate}
\section*{Introducci\'on} 
 
 (Xmin')

\textbf{(Como introducir mi clase?)}
		\begin{itemize}
			\item Recursos para motivar la clase.
			\item Recuento por los antecedentes de los resultados o investigadores.
			\item Esta no tiene que venir acompa�ada por plecas, solo es un ejemplo.
		\end{itemize}	
		

		
\section*{Teorizando un poco}	

(Ymin')

\begin{enumerate}
	\item \textbf{(Un Teorema interesante)} Tras la introducci\'on se podr\'an construir las secciones que se estimen convenientes para el desarrollo de la clase.  
	\item \textbf{(Un brillante algoritmo)} Los nombre de cada una de estas secciones quedan a la elecci\'on del autor.  

\end{enumerate} 



\section*{Ejercicio 1}

 (Zmin')
 
Compartir las soluciones de los ejercicios propuestos as\'i como observaciones interesantes que puedan servir de gu\'ia para el desarrollo de la clase.



\section*{Ejercicio 2} 

(20)

Muchas veces la interacci\'on con los estudiantes puede ser de importancia. De este modo posibles m\'etodos de soluci\'on variados pueden aportar al enriquecimiento del m\'etodo.

\textbf{Receso}(5')



\section*{Conclusiones}

 (Cierta cantidad de minutos')
 
Se resumir\'an los resultados m\'as destacados ejercitados en la actividad.

Se puede hacer menci\'on de aplicaciones del m\'etodo estudiado, posibles investigaciones o repercusiones en la cotidianidad. As\'i como los elementos de mayor significaci\'on. 



\section*{Estudio Independiente}

(Algun tiempo')

Orientar y comentar los ejercicios siguientes:

\begin{eje}
	De creerlo conveniente, la asignaci\'on de tareas para el estudio independiente, o la asignaci\'on de evaluaciones. 
\end{eje}
\begin{eje}
	 La cantidad de los mismos es a conveniencia aunque podr\'ia ser de ayuda su justificaci\'on.
\end{eje}

\section*{Ejercicio 3}

Para concluir, la soluci\'on de los ejercicios propuestos.



\section*{Ejercicio 4}
El esquema de clase es variable y queda sujeto a la voluntad del participante, lo que si deber\'a ajustarse a los requisitos de la convocatoria oficial.


  
\end{document}
